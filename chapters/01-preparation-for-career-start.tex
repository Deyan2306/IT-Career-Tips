\chapter{Preparation for Career Start -- How to Achieve a Successful Career Realization?}

This chapter covers the main concepts and ideas for starting your IT journey.

\begin{itemize}
    \item Career Centers
    \item  IT Recruitment Process
    \item Job Application Process
\end{itemize}

\section{Career Centers}
\textit{Career centers are usually a service, which guides you to finding the job, that you want, not the job that you need to start.}
They will help you find a job, fill in the documents and make the whole process way easier in general. Some of them may help you create a resume and cover letter as well. They will give you tips on crating an eye-catching portfolio. Some of them may even help you fill in information on social media like \textbf{LinkedIn}, and even give you technical assignment tests, so you could get prepared for the interview.
\section{IT Recruitment Process}
\textit{How Companies Hire New Employees?}
Companies usually hire following this process:
\begin{enumerate}
    \item Job Posting -- They will post a job for open position in any site \textit{(Like LinkedIn for example)}. There they give information about what an applicant should be able to do, as well as what they will be doing and what are they going to receive in return.
    \item Collect Applications -- In this step the companies should have to start collecting the applications from different people.
    \item Screening \& Selection -- Now is the HR's role - they have to go trough all of the applications and select the best ones.
    \item Contact Selected -- After the HRs select the candidates, they will contact them to let them know, that they have been selected for an interview. At this point they usually decide when the interviews would be held.
    \item HR Interview -- Here is the interview, where the IT companies will get to know who you are, how determined you are and what do you want to do.
    \item Technical Assessment -- After the HR interview, you have to pass a series of questions to let the company know, that you have some kind of knowledge.
    \item Job Offer -- After you are done with the Technical Assessment, you are sent a Job Offer, which tells you everything about the position. From here on you could decide if you want to start working for them or not.
\end{enumerate}

\subsection{What is the Process in Job Posting?}
The HR Department:
\begin{itemize}
    \item Defines the \textbf{job requirements} -- Job duties, Qualifications, Skills required.
    \item Detailed \textbf{job description} -- The HR department has to write a detailed description of the open position. \textit{They usually gather information from the previous step.} They add more information about the company, compensations and benefits.
    \item Choose the \textbf{right platforms} -- After the job description has been written, the HRs have to choose the right platform to post the job offer. It could be an online job board, company website or any social media. After that they publish the job, making sure it is \textit{well written} and \textit{easy to understand}. 
\end{itemize}

\subsection{Screening \& Selection}
Many candidates think, that this step is not related to them at all. But this is not true. If there is an open junior position \textit{(for example)} on the market, there are usually hundreds of applications. The HR has to go trough all of the CVs fast, so they do not lose any time. \textbf{It usually takes 6-7 seconds for the HR to go trough the job application.} That is why we have to make sure our application is as \textbf{eye-catching} as possible.

In this process HRs have to:
\begin{itemize}
    \item Review applications -- They have to determine which application \textbf{meets the basic requirements}. They have to shortlist the candidates. Because of the lack of time, they have to select only the \textbf{most suitable} candidates for an interview. This process is called \textit{Screening}. After your CV has been screened, you will have your resume and cover letter reviewed in more detail. In this steps you will have your LinkedIn or GitHub profile checked. 
\end{itemize}

\subsection{HR Interview}
The HR interview is low-key the most important step of the process. Here you will have to show the HR, that you have interest in the company. They will seek your determination, so make sure you show it. Present good communication skills and feel comfortable. And don't forget, that \textit{Interviews are a two-sided process}. Feel free to ask questions about the position.
In this step the company will start with a \textbf{short presentation} about yourself. They try to understand \textit{who} you are. They will ask if you have any previous experience \textit{(and if you have -- talk about it)}. After that they will evaluate your \textbf{tech skills}. HRs have some base tech skills. They are there just to make sure, that you really understand something, to make sure that you are suitable for Skills Assessment. They will want to know if the information that you filled in the CV is really true. They will want to test your behaviour as well, while testing your \textbf{problem solving} and \textbf{communication skills}.

Here is an example question: \textit{Imagine that you have a door, that is teleport. You have another door as well. You could put this door in any place in the world you want, so you could teleport to it in any time you want -- where would you place it?}

HR's job is to make sure that the workers in the company really work, so if you do not nail the HR interview, there is a big chance, that you will not go to a technical assessment interview. They might evaluate your \textit{soft skills} as well.
\subsection{Job Offer}
If selected, the candidate will receive a job offer.
The company will offer the candidate:
\begin{itemize}
    \item Salary to negotiate -- If you do not like the salary, you could talk about it with the company.
    \item Benefits -- Sports / Health insurance
    \item Start date
    \item Flexible working hours
\end{itemize}
\section{Job Application Process}
\textbf{Typical Steps to Apply for a Job and Get Hired}
The path to get your first tech job -- \textit{Developer}:
\begin{enumerate}
    \item Learn Coding and Technologies -- take your time and learn the stuff. They won't hire you if you know nothing.
    \item Create Projects Portfolio -- Have some projects to show off. Write them well, so they could get you hired.
    \item Research the Job Market -- Find what is searched now and fill in the gaps in your knowledge.
    \item Select a Job Position -- You should know \textit{where} to apply. \textit{Do you feel comfortable in this position?}
    \item Interviews
    \item Assessment
    \item Job Offer
    \item Career start
\end{enumerate}
\subsection{Extend Your Knowledge}
As the time goes on, new technologies are being created. So you would have to learn them somehow. Invest your time and do it. Nowadays it's free. You could pay for a course as well, where you would get a mentor, that will guide you. Here are example ways to learn a new technology:
\begin{itemize}
    \item Develop new skills through courses
    \item Watch online tutorials
    \item Find interesting books
    \item Mentorship programs
    \item Open source projects
    \item IT Career events
    \item Developers Community
\end{itemize}
\subsection{Get to Know with the Market Requirements}
You would have to select the offers, that you are really suitable for. If you lack some knowledge, do not start making thing up. Don't be shy to tell the HRs that you do not know something, but you are determined to learn it. Good English is required, but you are reading this, so it should not bother you. 
Find what is hot. Research the market and find the most searched technologies -- there is a bigger chance that you are going to find a job.
\subsection{Self-Preparation}
\textbf{Research the company that you are applying for!} See what you lack, see what they do, show them that you have interest in them! Find the most relevant company for yourself. In the next sections we are going to look at different types of companies, so you could understand if you want to work there.
\subsection{Document Preparation}
\begin{itemize}
    \item Resume -- Keep it short and customize it to match the job. Do not send the same CV to 20 different job offerings. Show your education, soft and technical skills, projects \textit{(with a description and link)}. Show off your GitHub profile. They look at code organization and commit history.
    \item Cover Letter
\end{itemize}
\section{Job Interview}
\textbf{A Few Important Preparation Steps}
The easiest way to prepare for an interview is to look up \textbf{frequently} asked interview questions on the technology. The companies would want to make sure what they do -- so look them up. Show specific examples of relevant experience. \textbf{Be honest about your skills!!} Show enthusiasm and interest. Be prepared to discuss salary expectations. Express gratitude for the chance and don't be late.
\section{Job Offer}
\textbf{What is inside?}
\subsection{In the Final Stage the Candidate may Discuss}
Seek opportunities for training or mentorship, so you could dip your toes faster. Discuss job responsibilities, relocation assistance, start date, salary, benefits, home office opportunity and working hours. Make sure that the offer suits you, but don't be way too picky ;) 
Good Luck.
